\documentclass{article}
\usepackage{lisp-on-tex}
\usepackage{qstest}
\IncludeTests{*}
\makeatletter
% test evaluation result
% #1 -> tevaluation target (LISP on TeX object form)
% #2 -> hoped result (LISP on TeX object form)
\def\testEval#1#2{%
  \def\@evaltarget{#1}%
  \setbox0=\hbox{\lispeval\@evaltarget\@evalresult}%
  \Expect*{\the\wd0}{0.0pt}% no side effect space
  \expandafter\checkObjectEquality\expandafter{\@evalresult}{#2}}
\def\testEvalSelf#1{\testEval{#1}{#1}}
% test evaluation result with local variables
% #1 -> tevaluation target (LISP on TeX object form)
% #2 -> hoped result (LISP on TeX object form)
% #3 -> environment (LISP on TeX environment form)
% NOTICE : this macro does NOT use \lispeval,
% so it ignores environment changing.
\def\testEvalWithEnv#1#2#3{%
  \setbox0=\hbox{\@eval#1{#3}\@evalresult}%
  \Expect*{\the\wd0}{0.0pt}% no side effect space
  \expandafter\checkObjectEquality\expandafter{\@evalresult}{#2}}
% test evaluation result with S-exp form
% #1 -> evaluation target (S-exp)
% #2 -> hoped result (LISP on TeX object form)
% NOTICE : this macro does NOT use \lispeval,
% so it ignores environment changing.
\def\testEvalWithSExp#1#2{%
  \@lispread\testEvalWithSExp@ReaderCallBack#1%
  \Expect*{\the\wd0}{0.0pt}% no side effect space
  \expandafter\checkObjectEquality\expandafter{\@evalresult}{#2}}
\def\testEvalWithSExp@ReaderCallBack#1#2{%
  \def\@eval@target{#1{#2}}%
  \setbox0=\hbox{\lispeval\@eval@target\@evalresult}}
% test equality of two object  
\def\checkObjectEquality#1#2{\@check@equal#1#2}
\def\@check@equal#1#2#3#4{%
  \ifx#1#3%
    \def\@check@next{%
      \ifx#1\@tlabel@cons
        \def\@check@next{\@check@equal@cons#2#4}%
      \else
        \def\@check@next{\Expect{#2}{#4}}%
      \fi
        \@check@next}%
  \else
    \def\@check@next{\Expect{#1}{#3}}%
  \fi
  \@check@next}
\def\@check@equal@cons#1#2#3#4{%
  \expandafter\expandafter\expandafter\@check@equal\expandafter#1#3%
  \expandafter\expandafter\expandafter\@check@equal\expandafter#2#4}
\makeatother
\begin{document}
  \LogTests{lgout}{*}{*}
  \makeatletter
  \begin{qstest}{Fundamental Eval Test}{\@lispeval}
   \begin{qstest}{self-evaluating forms}{\@lispeval}
    \testEvalSelf{\@tlabel@string{foo}} %string
    \testEvalSelf{\@tlabel@int{42}}     %integer
    \testEvalSelf{\@tlabel@bool{t}}     %boolean
    \testEvalSelf{\@tlabel@dimen{12.3pt}} %dimen
    \testEvalSelf{\@tlabel@skip{12.3pt plus 1.0 fil minus 0.5pt}} %skip
    \testEvalSelf{\@tlabel@nil{}}                                 % ()
    \testEvalSelf{\@tlabel@func{\dummy}}                          % function
    \testEvalSelf{%
       \@tlabel@closure{{\x:\@@unused}
         {\y{\@ylabel@int{1}}}\@tlabel@int{0}}}                    % closure
    \testEvalSelf{\@tlabel@macro{{\x:\@@unused}{}\@tlabel@int{0}}} % macro
    \testEvalSelf{\@tlabel@envfunc{\dummy}}                        % function form for apply and eval
    \testEvalSelf{\@tlabel@mutable{\dummy}}                        % mutable (used in deffileM and letM)
    \testEvalSelf{\@tlabel@lambda{}}                               % lambda special form 
    \testEvalSelf{\@tlabel@define{}}                               % define special form
    \testEvalSelf{\@tlabel@if{}}                                   % if special form
    \testEvalSelf{\@tlabel@defmacro{}}                             % defmacro special form
    \testEvalSelf{\@tlabel@begin{}}                                % begin special form
    \testEvalSelf{\@tlabel@mdefine{}}                              % defineM special form 
    \testEvalSelf{\@tlabel@setb{}}                                 % setB
    \testEvalSelf{\@tlabel@@let{}}                                 % @let special form (main function of let)
    \testEvalSelf{\@tlabel@@mlet{}}                                % @mlet special form (main function of letM)
    \testEvalSelf{\@tlabel@macroexpand{}}                          % macroexpand special form
    \testEvalSelf{\@tlabel@callocc{}}                              % callOCC special form
    \testEvalSelf{\@tlabel@exception{{-1}{\@tlabel@string{dummy}}}} % exception object 
    \testEvalSelf{\@tlabel@continuation{1}}                         % continuation object
   \end{qstest}
   \begin{qstest}{symbol}{\@lispeval,\@tlabel@symbol}
    \testEval{\@tlabel@symbol{\=}}
      {\@tlabel@func{\@lisp@func@check@args{2}{\@lisp@equal}}} % global binding
    \testEvalWithEnv{\@tlabel@symbol{\foo}}{\@tlabel@int{42}}
      {\hoge{\@tlabel@int{23}}\foo{\@tlabel@int{42}}} % local binding
    \def\dummyRef{\@tlabel@string{hoge}}
    \testEvalWithEnv{\@tlabel@symbol{\foo}}{\@tlabel@string{hoge}}
      {\foo{\@tlabel@mutable{\dummyRef}}} % mutable expansion
    \testEvalWithEnv{\@tlabel@symbol{\=}}{\@tlabel@string{hoge}}
      {\={\@tlabel@string{hoge}}} % global variable hiding
    \testEvalWithEnv{\@tlabel@symbol{\=}}{\@tlabel@int{42}}
      {\={\@tlabel@int{42}}\={\@tlabel@string{hoge}}} % local variable hiding
    % TODO unbound variable error test
   \end{qstest}
   \begin{qstest}{cons cell (function call)}{\@lispeval,\@tlabel@cons}
    \testEvalWithSExp{(\+ :1 :2)}{\@tlabel@int{3}}
    \testEvalWithSExp{(\+ :1 :2 . :3)}
      {\@tlabel@exception{{-1}{\@tlabel@string{Eval Improper List}}}}
   \end{qstest}
   \begin{qstest}{special forms}{\@lispeval}
    % define
    \toks0\expandafter{\@lisp@globalenv}
    \edef\@hopedEnv{\noexpand\@foo{\noexpand\@tlabel@int{42}}\the\toks0}
    \testEvalWithSExp{(\define \@foo :42)}{\@tlabel@nil{}}
    \toks0\expandafter{\@lisp@globalenv}
    \expandafter\Expect\expandafter{\@hopedEnv}*{\the\toks0}
    \testEvalWithSExp{(\begin (\define \@test@define@one :42) \@test@define@one)}
      {\@tlabel@int{42}}
    \testEvalWithSExp{\@test@define@one}
      {\@tlabel@int{42}}
    \lispinterp{(\@let \@foo :1 (\define \@test@define@local 'local'))}
    \testEvalWithSExp{\@test@define@local}{\@tlabel@string{local}}
    \lispinterp{((\lambda ()(\define \@test@define@local 'local2')))}
    \testEvalWithSExp{\@test@define@local}{\@tlabel@string{local2}}
    \testEvalWithSExp{(\define (\@foo \x \y) (\+ \x \y))}{\@tlabel@nil{}}
    \lispinterp{(\define (\@foo \x \y) (\+ \x \y))}
    \testEvalWithSExp{(\@foo :2 :3)}{\@tlabel@int{5}}
    \testEvalWithSExp{(\define :1 :2)}{\@tlabel@exception{{-1}
      {\@tlabel@string{The 1st argument of \define or \defineM must be a symbol or valid list.}}}}
    \testEvalWithSExp{(\define (:1) :2)}{\@tlabel@exception{{-1}
      {\@tlabel@string{The 1st argument of \define or \defineM must be a symbol or valid list.}}}}
    %defineM
    \toks0\expandafter{\@lisp@globalenv}
    \def\checkDefineMutableCheckEnv#1#2#3\@@end{%
      \Expect{#1}{\@bar}
      \checkDefineMutableCheckEnv@value#2
      \Expect{#3}*{\the\toks0}}
    \def\checkDefineMutableCheckEnv@value#1#2{%
      \Expect{#1}{\@tlabel@mutable}
      \expandafter\Expect\expandafter{#2}{\@tlabel@int{23}}}
    \testEvalWithSExp{(\defineM \@bar :23)}{\@tlabel@nil{}}
    \expandafter\checkDefineMutableCheckEnv\@lisp@globalenv\@@end
    \testEvalWithSExp{(\begin (\defineM \@test@defineM@one :42) \@test@defineM@one)}
      {\@tlabel@int{42}}
    \testEvalWithSExp{\@test@defineM@one}
      {\@tlabel@int{42}}
    \lispinterp{(\@let \@foo :1 (\defineM \@test@define@local 'local'))}
    \testEvalWithSExp{\@test@define@local}{\@tlabel@string{local}}
    \lispinterp{((\lambda ()(\defineM \@test@define@local 'local2')))}
    \testEvalWithSExp{\@test@define@local}{\@tlabel@string{local2}}
    \testEvalWithSExp{(\defineM (\@foo \x \y) (\+ \x \y))}{\@tlabel@nil{}}
    \lispinterp{(\defineM (\@foo \x \y) (\+ \x \y))}
    \testEvalWithSExp{(\@foo :2 :3)}{\@tlabel@int{5}}
    \testEvalWithSExp{(\defineM :1 :2)}{\@tlabel@exception{{-1}
      {\@tlabel@string{The 1st argument of \define or \defineM must be a symbol or valid list.}}}}
    \testEvalWithSExp{(\defineM (:1) :2)}{\@tlabel@exception{{-1}
      {\@tlabel@string{The 1st argument of \define or \defineM must be a symbol or valid list.}}}}
    %setB
    \lispinterp{
      (\defineM \@setB@target 'invalid')
      (\setB \@setB@target 'valid')
    }
    \testEval{\@tlabel@symbol{\@setB@target}}{\@tlabel@string{valid}}
    % if
    \testEvalWithSExp{(\lispif /t 'hoge' :42)}{\@tlabel@string{hoge}}
    \testEvalWithSExp{(\lispif /f 'hoge' :42)}{\@tlabel@int{42}}
    \testEvalWithSExp{(\lispif /x 'hoge' :42)}%
      {\@tlabel@exception{{-1}{\@tlabel@string{Fatal. Invalid Boolean}}}}
    \testEvalWithSExp{(\lispif :1 'hoge' :42)}%
      {\@tlabel@exception{{-1}{\@tlabel@string{The condition must be a bool.}}}}
    \lispinterp{(\defineM \@if@target 'valid')}
    \testEvalWithSExp{(\lispif /t 'hoge' (\setB \@if@target 'invalid'))}{\@tlabel@string{hoge}}
    \testEval{\@tlabel@symbol{\@if@target}}{\@tlabel@string{valid}}
    \testEvalWithSExp{(\lispif /f (\setB \@if@target 'invalid') :42)}{\@tlabel@int{42}}
    \testEval{\@tlabel@symbol{\@if@target}}{\@tlabel@string{valid}}
    % quote
    \def\@dummy@car{\@tlabel@int{1}}
    \def\@dummy@cdr{\@tlabel@nil{}}
    \testEvalWithSExp{(\quote (:1))}{\@tlabel@cons{\@dummy@car\@dummy@cdr}}
    % begin
    \lispinterp{(\defineM \beginSideEffect ())}
    \testEvalWithSExp{(\begin (\setB \beginSideEffect :1) 'hoge')}{\@tlabel@string{hoge}}
    \testEval{\@tlabel@symbol{\beginSideEffect}}{\@tlabel@int{1}}
    % lambda
    \testEvalWithSExp{(\lambda (\x \y) :42)}
      {\@tlabel@closure{{\x\y:\@@unused}{}\@tlabel@int{42}}} % argument list
    \testEvalWithSExp{(\lambda (\x.\y) :42)}
      {\@tlabel@closure{{\x:\y}{}\@tlabel@int{42}}} % argument pair
    \testEvalWithSExp{(\lambda \x :42)}
      {\@tlabel@closure{{:\x}{}\@tlabel@int{42}}} %argument variable
    \testEvalWithSExp{((\lambda (\x) (\lambda (\y) :42)) :1)}
      {\@tlabel@closure{{\y:\@@unused}{\x{\@tlabel@int{1}}}\@tlabel@int{42}}} %environment
    \lispinterp{ % free valiable's test. If the language takes lexical
                 % scope, this code raise exception. 
      (\define \testLocalness (\lambda () \unbound))
      (\defineM \@valtest :42)
      (\let ((\unbound 'invalid')) (\setB \@valtest (\testLocalness)))}    
    \testEvalWithSExp{(\let ((\unbound 'invalid')) (\setB \@valtest (\testLocalness)))}
      {\@tlabel@exception{{-1}{\@tlabel@string{unbound variable \unbound}}}}
    \testEvalWithSExp{\@valtest}{\@tlabel@int{42}}
    \testEvalWithSExp{((\lambda \x \x))}{\@tlabel@nil{}} % non argument remain test  
    %@let
    \lispinterp{(\define \@letTarget 'invalid')}
    \testEvalWithSExp{(\@let \@letTarget 'valid' \@letTarget)}{\@tlabel@string{valid}}
    \testEval{\@tlabel@symbol{\@letTarget}}{\@tlabel@string{invalid}}
    %@mlet
    \lispinterp{
      (\defineM \@mletTarget 'invalid')
      (\defineM \@mletTemp 'hoge')
    }
    \testEvalWithSExp{(\@mlet \@mletTarget 'valid' 
      (\begin (\setB \@mletTemp \@mletTarget) (\setB \@mletTarget 'modified') \@mletTarget))}
      {\@tlabel@string{modified}}
    \testEval{\@tlabel@symbol{\@mletTemp}}{\@tlabel@string{valid}}
    \testEval{\@tlabel@symbol{\@mletTarget}}{\@tlabel@string{invalid}}
    % defmacro
    \lispinterp{(\defmacro \defmacroTarget (\lambda (\c) \c))}
    \testEval{\@tlabel@symbol{\defmacroTarget}}
      {\@tlabel@macro{{\c:\@@unused}{}\@tlabel@symbol{\c}}}
   \end{qstest}
   \begin{qstest}{one-shot continuation}{\@lispeval}
    \testEvalWithSExp{(\callOCC (\lambda (\c) :42))}{\@tlabel@int{42}} % normal end
    \testEvalWithSExp{(\callOCC (\lambda (\c) (\callOCC (\lambda (\d) \d))))}
      {\@tlabel@continuation{2}} %return continuation itself
    \testEvalWithSExp{(\callOCC (\lambda (\c) (\callOCC (\lambda (\d) \d))))}
      {\@tlabel@continuation{2}} %return continuation itself (renumber check)
    % works on \begin
    \lispinterp{(\defineM \@tmp 'invalid')}
    \testEvalWithSExp{(\callOCC (\lambda (\c) 
      (\begin (\setB \@tmp :1) (\c :2) (\setB \@tmp :-1))))}{\@tlabel@int{2}}
    \testEval{\@tlabel@symbol{\@tmp}}{\@tlabel@int{1}}
    % works on argument evaluation (function)
    \lispinterp{(\setB \@tmp 'valid')}
    \testEvalWithSExp{(\callOCC (\lambda (\c) (\setB \@tmp (\c :42))))}
      {\@tlabel@int{42}}
    \testEval{\@tlabel@symbol{\@tmp}}{\@tlabel@string{valid}}
    % works on argument evaluation (environment function)
    \def\@dummy@car{\@tlabel@int{1}}
    \def\@dummy@cdr{\@tlabel@int{2}}
    \testEvalWithSExp{(\callOCC (\lambda (\c) (\eval (\c (\quote (:1.:2))))))}
      {\@tlabel@cons{\@dummy@car\@dummy@cdr}}
    % works on argument evaluation (closure)
    \testEvalWithSExp{(\callOCC (\lambda (\c) ((\lambda (\arg) 'invalid') (\c 'valid'))))}
      {\@tlabel@string{valid}}
   \end{qstest}
   \begin{qstest}{closure evaluation}{\@lispeval}
    % non remain argument
    \testEvalWithSExp{((\lambda (\x \y) (\+ \x \y)) :1 :2)}{\@tlabel@int{3}}
    % remain argument
    \def\@dummy@car{\@tlabel@int{2}}
    \def\@dummy@cdr{\@tlabel@nil{}}
    \testEvalWithSExp{((\lambda (\x . \y) \y) :1 :2)}{\@tlabel@cons{\@dummy@car\@dummy@cdr}}
    % all argument list
    \def\@dummy@car{\@tlabel@int{1}}
    \def\@dummy@cdr{\@tlabel@nil{}}
    \testEvalWithSExp{((\lambda \x \x) :1)}{\@tlabel@cons{\@dummy@car\@dummy@cdr}}
   \end{qstest}
   \begin{qstest}{macro evaluation and macroexpand}{\@lispeval}
    \lispinterp{(\defmacro \mac (\lambda (\x) ((\lambda \y \y) (\quote \+) \x \x)))}
    \testEvalWithSExp{(\mac :1)}{\@tlabel@int{2}}
    \def\@dummy@car{\@tlabel@symbol{\+}}
    \def\@dummy@cdr{\@tlabel@cons{\@dummy@car@ii\@dummy@cdr@ii}}
    \def\@dummy@car@ii{\@tlabel@int{1}}
    \def\@dummy@cdr@ii{\@tlabel@cons{\@dummy@car@iii\@dummy@cdr@iii}}
    \let\@dummy@car@iii\@dummy@car@ii
    \def\@dummy@cdr@iii{\@tlabel@nil{}}
    \testEvalWithSExp{(\macroexpand (\quote (\mac :1)))}{\@tlabel@cons{\@dummy@car\@dummy@cdr}}
   \end{qstest}
   \begin{qstest}{(environment) function evaluation}{\@lispeval}
    \testEvalWithSExp{(\+ :1 :2)}{\@tlabel@int{3}} % function
    \testEvalWithSExp{(\apply \+ (\quote (:1 :2)))}{\@tlabel@int{3}} % apply (env func)
    \lispinterp{(\define \@tmp 'valid')}
    \testEvalWithSExp{(\eval (\quote \@tmp))}{\@tlabel@string{valid}} % eval (env func)
   \end{qstest}
   \begin{qstest}{function arguments check}{\@lispeval}
    \def\hoge#1#2\relax\relax{\gdef#1{\@tlabel@string{called}}}
    % * case
    \@lisp@env@add@global\hoge{\@tlabel@func{\@lisp@func@check@args{*}{\hoge}}}
    \tracingmacros=1
    \testEvalWithSExp{(\hoge)}{\@tlabel@string{called}}
    \testEvalWithSExp{(\hoge :1)}{\@tlabel@string{called}}
    \testEvalWithSExp{(\hoge :1 :3)}{\@tlabel@string{called}}
    % + case
    \@lisp@env@add@global\hoge{\@tlabel@func{\@lisp@func@check@args{+}{\hoge}}}
    \testEvalWithSExp{(\hoge)}
     {\@tlabel@exception{{-1}{\@tlabel@string{too few arguments}}}}
    \testEvalWithSExp{(\hoge :1)}{\@tlabel@string{called}}
    \testEvalWithSExp{(\hoge :1 :3)}{\@tlabel@string{called}}
    % n=2 case
        % * case
    \@lisp@env@add@global\hoge{\@tlabel@func{\@lisp@func@check@args{2}{\hoge}}}
    \testEvalWithSExp{(\hoge)}
     {\@tlabel@exception{{-1}{\@tlabel@string{too few arguments}}}}
    \testEvalWithSExp{(\hoge :1)}
     {\@tlabel@exception{{-1}{\@tlabel@string{too few arguments}}}}
    \testEvalWithSExp{(\hoge :1 :3)}{\@tlabel@string{called}}
    \testEvalWithSExp{(\hoge :1 :3 :2)}
     {\@tlabel@exception{{-1}{\@tlabel@string{too many arguments}}}}
   \end{qstest}
  \end{qstest}
  \LogClose{lgout}
\end{document}
