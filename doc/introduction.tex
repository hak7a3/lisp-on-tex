\section{Introduction}
LISP on \TeX{} is a \LaTeX{} class to run 
LISP programs in a document.
All of it is written with \TeX{} macros, so
we do not need a special \TeX{} engine,
\texttt{\string\write18}, and external language systems.
LISP on \TeX{} works if you put its all style files
to your \texttt{texmf} tree. 

\subsection{Getting Started}
In order to use LISP on \TeX{}, you should load
\texttt{lisp-on-tex} package: write
\begin{quote}
\begin{verbatim}
\usepackage{lisp-on-tex}
\end{verbatim}
\end{quote}
on your document's preamble.
Then, you can execute LISP codes by
\texttt{\string\lispinterp}. For example, the code
\begin{quote}
\begin{verbatim}
\lispinterp {
  % define \succ function.
  (\define (\succ \n) (\+ \n :1))
  % call \succ and print the result.
  (\texprint (\succ :42))
}
\end{verbatim}
\end{quote}
outputs ``%
\lispinterp {
  (\define (\succ \n) (\+ \n :1))
  (\texprint (\succ :42))
}%
''.
As you can see in the example, you can use \texttt{\%} as
starting comment.  
The \texttt{\string\lispinterp} is not \texttt{\string\long}ed, so
you CANNOT include empty lines into a LISP on \TeX's program.

\subsection{Class Options}
LISP on \TeX{} has options for garbage collection (GC).
If you want to use GC, use \texttt{markGC} option.
You can also assign heap size by
\texttt{GCopt=\{heapsize=\textrm{$n$}\}} where $n$
is an integer. The default heap size is 32768.
For example, the code
\begin{quote}
\begin{verbatim}
\usepackage[markGC, GCopt={heapsize=40000}]{lisp-on-tex}
\end{verbatim}
\end{quote}
means LISP on \TeX{} uses GC and the heap size is 40000.
