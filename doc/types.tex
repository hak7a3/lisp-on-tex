\section{Objects}
We define LISP on \TeX's objects by using a grammatical notation
like the \TeX book. In this section, \nonterm{foo} is a non terminal symbol,
\term{bar} is a terminal symbol, $\define$ means ``is defined to be,''
and $\bnfor$ means ``or''. The operator $*$ is Kleene star, and $+$
is Kleene plus.

\subsection{Integers}
An integer is \nonterm{integer}:
\[
 \nonterm{integer} \define \term{:}\nonterm{\TeX's number}
\]
where \nonterm{\TeX's number} is \nonterm{number} in the \TeX book.
For example, \term{:-42} means $\lispinterp{(\texprint :-42)}$, 
\term{:"BEEF} means $\lispinterp{(\texprint :"BEEF)}$, and
\term{:`\string\@} means $\lispinterp{(\texprint :`\@)}$.

\subsection{Strings}
An string is an \TeX's \nonterm{balanced tokens} surrounded by \term{'}:
\[
 \nonterm{string} \define \term{'}\nonterm{balanced tokens}\term{'}
\]
If you want to include \term{'}, you should use brace; the code
\term{'\{`{}`\}quoted \string\TeX\{\} tokens\{'{}'\}'} means 
\lispinterp{(\texprint '{``}quoted \TeX{} tokens{''}')}.
In ordinary Lisp interpretation, \term{'} is used for abbreviation
of \term{quote}. In contrast, LISP on \TeX{} does not support it.

\subsection{CONS cells and nil}
A CONS cell is \nonterm{cons cell} and the value nil is \nonterm{nil}:

\begin{eqnarray*}
  \nonterm{cons cell} &\define& \nonterm{proper list} \bnfor
   \nonterm{improper list} \\
  \nonterm{proper list} &\define& \term{(} \nonterm{object}\mathop{+} \term{)} \\
  \nonterm{improper list} &\define& \term{(} \nonterm{object}\mathop{+}
   \ \term{.}\ \nonterm{object}\mathop{+} \term{)}\\
  \nonterm{nil} &\define& \term{()}
\end{eqnarray*}
where \nonterm{object} is a LISP on \TeX's object.

\subsection{Symbols}
In LISP on \TeX, a symbol is a control sequence.
For example, \term{\string\somecs} is a symbol.

\subsection{Booleans}
A boolean is \nonterm{bool};
\[
 \nonterm{bool} \define \term{/t} \bnfor \term{/f}
\]
The term \term{/t} means true, and \term{/f} means false.

\subsection{{\manual \char127}Reserved Forms}